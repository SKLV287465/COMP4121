\documentclass[12pt, letterpaper]{article}
\usepackage{graphicx}
\usepackage{amsmath}
\title{Database Access}
\author{Joshua Shim}
\date{October 2024}

\begin{document}
\maketitle
\newpage
\textbf{Database Access Problem:}\\
\begin{itemize}
    \item Assume that $n$ processes want to acces a database, and that the time is discrete, $t = 1, 2, 3,\dots$
    \item Simultaneous access requests $\rightarrow$ conflict and all proccesses are locked out of access.
    \item We assume that the processes cannot communicate with each other to agree on a joint startegy.
\end{itemize}
\textbf{Solution:}\\
One possible approach in such a situation is that each process at each instant $t$ requests with probability $p$ and does not request with probability $1 - p$. How should we choose $p$ to maximise the probability of a successful access to the database for a process at any instant $t$?
\begin{enumerate}
    \item Access Success Probability: $P(S(i, t)) = p(1-p)^{n - 1}$
    \item Extremal points found by: \[\frac{d}{dp}P(S(i, t)) = (1 - p)^{n - 1} - p(n - 1)(1 - p)^{n - 2} = 0\]
    \item Dividing both sides by $(1-p)^{n-2}$, we get that the above equality holds just in case $1-p-p(n-1) = 0$ which is equivalent to $p = \frac{1}{n}$.
    \item Hence we get the the optimal $p = \frac{1}{n}$. Hence \[P(S(i, t)) = p(1-p)^{n - 1} = \frac{1}{n}(1 - \frac{1}{n})^{n - 1}\]
    \item The following two facts are useful: \begin{itemize}
        \item $(1-\frac{1}{n})^n$ increases monotonomically from $\frac{1}{4}$ up to $\frac{1}{e}$ as $n$ increases from 2 to $\infty$.
        \item $(1-\frac{1}{n})^{n - 1}$ decreases monotonically from $\frac{1}{2}$ down to $\frac{1}{e}$ as $n$ increases from 2 to $\infty$.
    \end{itemize}
    \item Since we had \[P(S(i, t)) = \frac{1}{n}(1 - \frac{1}{n})^{n - 1}\] we obtain \[\frac{1}{n}\cdot\frac{1}{e}\leq P(S(i, t))\leq \frac{1}{n}\cdot\frac{1}{2}\]
    \item Thus $P(S(i, t)) = \Theta(\frac{1}{n})$.
    \item $P(\text{failure after $t$ instants}) = (1 - \frac{1}{n}(1 - \frac{1}{n})^{n - 1})^t$.
    \item Using the second inequality we get $P(\text{failure after $t$ instants}) \approx (1 - \frac{1}{en})^t$.
    \item \underline{Strange phenomenon:} If we choose $t = en$ many consecutive instants, then the probability of failure is quite large, because \[P(\text{failure after $t = en$ instants}\approx (1 - \frac{1}{en})^{en}\frac{1}{e})\]
    \item However, if we increase the number of instants only slightly , by taking $t = en\cdot 2\ln(n)$, then \begin{align*}
        P(\text{failure after $t = en2\ln(n)$ instants}) &\approx (1 - \frac{1}{en})^{2n2\ln(n)}\\
        &= ((1-\frac{1}{en})^{en})^{\ln(n^2)}\\
        &\approx (\frac{1}{e})^{\ln(n^2)}\\
        &=\frac{1}{n^2}
    \end{align*}
    \item Thus a slight increase in the number of time instants from $en$ to $2dn\ln(n)$ caused a dramatic reduction in the probability of failure.
    \item If failure probability is less than $\frac{1}{n^2}$ and there are $n$ processes, then probability that at least one process failed cannot be larger than $n \cdot \frac{1}{n^2} = \frac{1}{n}$.
    \item Thus after $2en\ln(n)$ instants, all processes succeeded to access the db with probability of at least $1-\frac{1}{n}$.
\end{enumerate}
\end{document}